
\begin{appendices}

\section{Binary Syntax}\label{binary_syntax}
The binary serialization of Circuit-IR will be described here using the \href{https://google.github.io/flatbuffers/}{open-source FlatBuffers cross-platform serialization library}, originally developed by Google.
FlatBuffers is a metaformat that specifies the superficial aspects of the syntax, such as representations of literals, structured data and arrays.
It moreover supports formal schemas that concretely define what elements (e.g., structures and arrays) can appear in the specific format.
FlatBuffers was chosen for the following reasons:
\begin{itemize}
\item It offers an existing compact encoding of the format with efficient (de)serialization.
\item It is supported by a wide-range of community-based tools and libraries for the most common languages (and is also easy to parse from scratch).
\end{itemize}

We will use the below FlatBuffers schema to represent \texttt{circuit}, \texttt{public\_input}, and \texttt{private\_input} ressources. This schema is isomorphic to the Circuit-IR representation presented in Section~\ref{sec:circuitir}.

\lstinputlisting[language=flatbuffer]{../sieve_ir.fbs}

As a structured format, the FlatBuffers schema provides a concrete, readable and typed syntax,
ensuring syntactic validity.
However, it does not provide resource or evaluation validity as it is not a language.
Refer to Section~\ref{circuit_ir_validity} to a description of syntactic, resource and evaluation validity.

A limitation of the Flatbuffer technology is its 32-bit internal pointer representation, which prevents it from storing buffers larger than approximately 2GB.
The IR specifies the following workaround for this limitation.

\begin{itemize}
  \item The \texttt{Root} message may be repeated within a file or stream as manytimes as is necessary: each message holding a portion of the IR resource.
  \item Each message must be prefixed by its length in bytes, as a 4-byte unsigned little-endian number. (See \href{https://google.github.io/flatbuffers/class_flat_buffers_1_1_flat_buffer_builder.html#a425ab2bd13a0e4331a7190ec2d17c3b2}{FinishSizePrefix}).
  \item Each message's \texttt{version} attribute must be the same as the first message's \texttt{version}. All other attributes must be empty except for the resource's body.
\end{itemize}

Unfortunately, there is no way for a Flatbuffer to hold a single function which is larger than 2GB.

\end{appendices}
