\section{Introduction}\label{sec:intro}

In Phase II of the SIEVE Program, we are currently defining two Intermediate Representations
called Circuit IR and Translation IR and a Circuit Configuration Communication file.
This document will only present the current version of Circuit IR.

The IR is an interaction between the frontend and the backend portions of a Zero Knowledge (ZK) proof pipeline.
The frontend transforms high level statements in a target domain into the IR.
It is the producer of the three resources which are the subject of this document.
The backend is the consumer of them: it is an interaction between a Prover and a Verifier, where the Prover wishes to prove a statement to the Verifier without revealing a secret component of the proof.

\subsection{Overview}

At a high level, this interaction involves three resources provided by the frontend and used by the backend, as the fact is being proven. \\

\begin{itemize}
  \item Relation (Circuit or Translation IR) -- a mathematical relationship between the inputs.
  \item Public inputs -- inputs to the relation given to both the Prover and Verifier.
  \item Private inputs -- inputs to the relation given only to the Prover.
\end{itemize}

Version 2 of the SIEVE IR seeks to resolve the conflict between the necessity for frontend and backend interoperability and the variety of design choices, capabilities, and requirements in each system.
The SIEVE program has seen everything from C++ libraries through R1CS as viable backend-specific IRs.

To resolve this issue, the SIEVE IR introduces two types of relation: the Circuit-IR and the Translation-IR.
\begin{itemize}
\item The Circuit-IR is defined by flat lists of gates and wires; functions may be defined and reused within the circuit.
\item The Translation-IR is a program which outputs a relation in the Circuit-IR format.
The backend is given some amount of control over how the Translation-IR is translated, and is free to reimplement common or even standardized libraries.
\end{itemize}

At QEDIT, we decided to focus on the Circuit IR for the following reasons 
\begin{itemize}
\item it is easily plugable to frontends/backends because it is close to existing representations (e.g. R1CS, PLONK)
\item it supports complex features and it is easily extensible thanks to plugins
\item it is a simple representation that simplifies newcomers' adoption and maintainability
\end{itemize}


\subsection{Multi Field Circuits}
To most practitioners of ZK, a single prime field is chosen at the beginning of a proof and used throughout.
However, for some applications, it is desirable to use multiple primes for different elements within a single larger proof.
For example, a large and expensive prime may be needed to verify public-key signatures, while a medium sized prime is necessary for large scale business logic.

To accommodate these applications, the IR must allow for multiple fields within a single relation.
To the frontend, each field must describe the type of a wire, while to the backend, these wires actually belong to multiple independent proofs.
An analogy to the real world might be a circuit card with transistor logic on one side and high voltage on the other.

Occasionally information from one field will be required in another.
The IR models this using a conversion gate with inputs in one field and outputs in another.
To continue the analogy, a relay would allow information to flow from transistor logic into high voltage, or in reverse, an analog-digital converter.
In ZK, methodologies must be developed and used to show the equivalence of inputs and outputs across independent proofs or even across different proof systems.

